\documentclass[a4paper,12pt]{report}

\usepackage{graphicx}
\DeclareGraphicsExtensions{.pdf,.png,.jpg}

\usepackage{wrapfig}

\author{Philip Peder Hansen}
\title{Visualisering af funktioner af to variable i Minecraft}

\begin{document}
	\maketitle
	\tableofcontents
	\clearpage
	\section{Funktioner af to variable}
	\subsection{Afbildning af funktioner af to variable}
	\section{Minecraft}
		Minecraft er et videospil lavel af Markus "Notch" Persson.
		Spillet er en blanding mellem et eventyr overlevelses spil, og et kreativt spil der g\aa r ud p\aa \ at bygge verdener.

		\vspace{5 mm}
		\includegraphics[keepaspectratio = true, scale = 0.2]{screenshot1}
		\vspace{5 mm}

		Indledene bliver spilleren placeret i en stor verden, der best\aa r af 1 $m^3$ blokke. Disse blokke har forskellige fysiske egenskaber,
		og bruges i forskellige sammenh\ae nge.

		Derudover genereres blokkene baseret p\aa \ nogle algorytmer, der resultere i et m\o nster der til nogen grad minder
		og den virkelige verden. Tr\ae r genereres i n\ae rheden af hinanden, i skove, og ikke midt ude i havene.
		\O rkner og tundra findes generelt ikke umidelbart i n\ae rheden af hinanden, og floder l\o ber ofte gennem regnskove. 
	\subsection{Generering af verdner}
		Algorytmerne der bruges til at generere verdener i Minecraft er en del mere komplicerede end en det simple eksemepel p\aa \ den funktion
		af to parametre vi har kigget p\aa . Ud over $X$ og $Z$ koordinated bruger minecraft også noget som kaldes et seed til at generere
		verdener, de genereres alts\aa \ ud fra en mere kompliceret funktion af tre parametre.
		
		\begin{wrapfigure}{r}{0.5\textwidth}
			\includegraphics[keepaspectratio = true, width = 0.5\textwidth]{screenshot2}
			\caption{Minecraft verden genererings sk\ae rm}
		\end{wrapfigure}

		Et seed er en tekst streng som brugeren kan give Minecraft n\aa r en ny verden genereres, hvis brugeren ikke giver denne streng
		bliver den automatisk genereret f\o r verdenen laves.
	\subsection{Planlægning}
	\subsection{Implementering}
	\subsection{Vurdering}
	\section{Perspektivering}
\end{document}